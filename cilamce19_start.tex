\documentclass{cilamce19}

\begin{document}
%Insert the header


%%This is the short title that will be inside the header 
\shorttitle{A Synthetic Approach for a Single Rock Detection using SLP.}
 
\begin{titlepage}
	
	\newgeometry{top=1.3in,bottom=0.8in,right=0in,left=2in}

	
	\begin{figure}
	\flushright
	\includegraphics[width=2.25in]{clm19_lb.png}
    \end{figure}


 \begin{center}
 	\begin{title}
 		\centering
 		\textbf{A Synthetic Approach for a Single Rock Detection using Single Layer Perceptron for Well Logging Data.}
 	\end{title}	
 \end{center}



%Authors and Affiliations
\noindent
\textbf{Victor R. Carreira$^{1}$}
\\
\textbf{Cosme F. Ponte-Neto$^{1}$}
\\
\textit{victorcarreira@on.br}\\
\textit{cosme@on.br}
\\
\textbf{Observat\'orio Nacional$^{1}$}
\\
\textit{Gen. Jos\'e Cristino Street, 77, 20921-400, Rio de Janeiro, Brazil}
\\
\textbf{Rodrigo S. Bijani$^{2}$}
\\
\textit{rodrigobijani@gmail.com}\\
\textbf{Universidade Federal Fluminense$^{2}$}
\\
\textit{Av. Gen. Milton Tavares de Souza, s/n, 24210-346, Niter\'oi, Brazil}
\\

%Abstract
\noindent %Abstract should not be indented
\textbf{Abstract:}   Machine learning approaches the creation of computer programs that have the capability of automatically improve them selves through time. single-layer perceptron (SLP) is the key elementary component in multilayer feed-forward networks used to solve real-world problems. The behavior of a single neuron serves as the simplest prototype model for studying the characteristics of more general non-linear models such as multilayer perceptrons. This work aims to define a methodology that are capable of indicate if a well logging data for a single rock horizons is a linear separable problem. Sedimentary rocks reflects the ancients environments in which they are formed, in which some of have their attention among petroleum industry. Some rocks have special signatures concerning well logging data. Linear discriminator analysis (LDA) is used as a constrain  validation to determine lithotype is separable from others groups of rocks. LDA analysis indicates a recovery data about $62.5\%$ concerning shale lithotype class. In the synthetic log data presented in this work, it is mentionable that SLPs are not able to define the shale dolomite contact. This result depicts that the resistivity should not be considered as a promising classification of lithotype. Additionally, we could observe that the bottom of the shale is not well defined, which indicates a resolution problem to be faced in future investigations.

\noindent %keywords should not be indented
%Keywords
\textbf{Keywords:} First keyword, Second keyword, Third keyword (up to 5 keywords) 

\end{titlepage}




%Here your work really starts.
\section{Introduction}
   Artificial neural networks (ANN) are elaborated machine learning algorithm inspired by human brain \citet{Hagan1996}. Therefore an ANN is a mathematical algorithm that simulates tasks performed by a single neuron or a group of neurons. A neuron is the basic unit of an ANN \citet{Nedjah2016}. In a natural neural network, information pass through electrical impulses transmitted by synapses \citet{Krogh2008}.  In literature, several ANNs are proposed to solve problems of different disciplines \citet{Al-Anazi2010,Bruyelle2014,Adibifard2014,Kumar2015}. This work uses a special kind of ANN know as Single Layer Perceptron (SLP). 
   
   The specific problem of logging wells, an important step is the identification of top and base layers that may be associated with changes in petrophysical properties \citet{Saljooghi2014}. Algorithms based on derivatives in the log curves do not identify very thin, or noisy layers \citet{Zhang1999}. \citet{Chakravarthy1999} can use the radial function to locate high-definition layer boundaries in induction log data (HDIL). However, \citet{Benaouda1999} can classify lithologic types into partially collapsed wells through the use of the neural network with error propagation and class changes as the analysis proceeds \citet{Poulton2002}. \citet{Gloaguen2017} raises the question of the relative importance of physical properties in well profiling data for neural network decision making.
   
   A single Perceptron only solve linear problems. This work aims to define a linear separation methodology for a single problem of shale detections. Shales are important lithotype units. By definition is a sedimentary rock composed by silt and clay minerals that have a detritial origin.  Color variations are important to define the specificity of such rock components (i.e., from dark shale rich in organic matter, gray that indicates a reducing environment in which the rock was deposited, or red-shale indicative of ferric oxide) .They bring information about reductive or oxidize environments. That analysis  could bring important informations concerning petroleum horizons. For oil and gas industry shale could be a generator rock (conventional petroleum system) or a reservoir and a generator rock  in a non-conventional perspective. 
   
   
   In this work, we define the following method for identification of shale horizons. Firstly, we implement a SLP in Fortran-90 framework considering some particular aspects, like two different learning rate calculations could be performed under this program the classical approach defined by \citet{Robbins1951} where the learning rate parameter is time dependent. Another approach is used by \citet{Darken1990}. Where  is defined a mechanism of search and convergence in learning rate calculation. 

\subsection{Synthetic Sedimentary Basin}

The proposed model for the machine learning tests was based on a schematic geological model proposed by  \citet{Sal2008} for the Solim\~oes Sedimentary Basin, North part of Brazil. This modeling reproduces structures such as Horst, Grabens, normal and reverse faults. Fig. \ref{t1} shows training well data. Four physical data properties were considered: density, gamma-ray, resistivity and velocity. The sample rate for the well data is $0.01$ observation/meter with contamination of $5$\% Gaussian noise.

\section{Methodology}

In a general overview, the methodology adopted in this work is divided into three main parts. The first generates a synthetic syneclises sedimentary basin in which synthetic wells are drilled. The second part uses well log T$1$  (see Fig. \ref{t1}) to LDA analysis and training the SLP to obtain an optimal distribution of weights. Additionally, the same well is again used to store T$1$ log data into arrays. The last is to use SLP and compare the classified shale patterns for  wells C$1$.

\begin{center}
%	\begin{tikzpicture}
%	[node distance=.8cm,
%	start chain=going below,]
%	\node[punktchain, join] (investeringer) {Input};
%	\node[punktchain, join] (intro) {Preprocessing};
%	\node[punktchain, join] (probf) {Linear Discriminator Analysis - LDA};
%	\node[punktchain, join] (investeringer) {Perceptron - SLP};
%	\node[punktchain, join] (investeringer) {Output};
%	\end{tikzpicture}

\usesmartdiagramlibrary{additions}
\smartdiagramset{%uniform color list=white!60!black for 6 items,
	back arrow disabled=true, module minimum width=1.8cm,
	module minimum height=2cm,
	module x sep=2.7cm,
	text width=1.8cm,
	additions={
		additional item offset=0.5cm,
		additional item width=2cm,
		additional item height=2cm,
		additional item text width=3cm
	}
}

\smartdiagram[flow diagram:horizontal]{Input, Preprocessing,  LDA,  SLP, Output} 
\end{center}


\subsection{Preprocessing}

In terms of neural networks a rock is defined as a mathematical function of it physical properties. A shale is defined according to Eq. \ref{shale}.

\begin{eqnarray}
\textbf{S}(\rho, \gamma, \Omega,\upsilon )
\label{shale}
\end{eqnarray}

Where $\rho, \gamma, \Omega,\upsilon $ are density ($g/cm^{3}$), gamma-ray ($Ci/g$), resistivity ($Ohm.m$) and velocity ($km/s$). The problem is a four degree space dimension. 

From a training well T$1$ (\ref{t1}), is formulate two data sets. One composed only by shale and the second with other rocks presents in T$1$ well. Be $\textbf{R}$ a set of $n$ vectors patterns that defines a rock in a fourth degree dimensional space. $\textbf{R} = \left\lbrace \textbf{R}_{i}(\vec{\mu}) \middle| i = 1, 2, 3, 4, \dots, n \right\rbrace $. $\textbf{R}_{1}$ and $\textbf{R}_{2}$ are binary partition ($\textbf{R}_{1}$, $\textbf{R}_{2}$ $\in$ $\textbf{R}$) of each point measurement in well data. 

\begin{multicols}{2}
	\begin{equation}
	\textbf{$\textbf{R}_{1}$} =
	\begin{vmatrix} 
	\xi_{11} & \xi_{12} & \xi_{13} & \xi_{14} \\
	\xi_{21} & \xi_{22} & \xi_{23} & \xi_{24} \\
	\xi_{31} & \xi_{32} & \xi_{33} & \xi_{34} \\
	\vdots   & \vdots   & \vdots   & \vdots   \\
	\xi_{m1} & \xi_{m2} & \xi_{m3} & \xi_{m4}
	\end{vmatrix}
	\nonumber
	\end{equation}
	\hspace{100cm}
	\begin{equation}
	\textbf{R}_{2}=
	\begin{vmatrix} 
	\zeta_{11} & \zeta_{12} & \zeta_{13} & \zeta_{14} \\
	\zeta_{21} & \zeta_{22} & \zeta_{23} & \zeta_{24} \\
	\zeta_{31} & \zeta_{32} & \zeta_{33} & \zeta_{34} \\
	\vdots   & \vdots   & \vdots   & \vdots   \\
	\zeta_{m1} & \zeta_{m2} & \zeta_{m3} & \zeta_{m4}
	\end{vmatrix}
	\nonumber
	\end{equation}
\end{multicols}   

A binary partition $\textbf{R}$ ($\textbf{R}_{1}$,$\textbf{R}_{2}$) is a dichotomy if in the group of surface families exists one hyperplane that separates the points of the shale class $\textbf{R}_{1}$ \footnote{$\textbf{R}_{1}$ $\in$ $\textbf{S}$ | $\exists$ $\textbf{S}$($\rho, \gamma, \Omega, \upsilon$) $\in$ $\Re$  } from the majority of points of non-shale class $\textbf{R}_{2}$.


\subsection{Linear Discriminator Analysis}

Linear discriminant analysis introduced by \cite{Cover1965} is a known dimension reduction and classification approach that has received much attention in the statistical literature. As such different versions of classification procedures have been introduced for various applications \citep{Okwonu2012}

If the dichotomy $\textbf{R}$ ($\textbf{R}_{1}$,$\textbf{R}_{2}$) exists there must be an n-dimensional vector $\vec{\omega}$ which can be written as relations described in Eq. \ref{cond1} and Eq. \ref{cond2} as follows.

\begin{eqnarray}
\vec{\omega}^{T}\textbf{R}(\vec{\mu}) > 0, \mu \in \textbf{R}_{1}
\label{cond1}
\end{eqnarray}

\begin{eqnarray}
\vec{\omega}^{T}\textbf{R}(\vec{\mu}) < 0, \mu \in \textbf{R}_{2}
\label{cond2}
\end{eqnarray}

If the relations conceived in Eq. \ref{cond1} and Eq. \ref{cond2} exists a class separator hyperplane that can be defined as Eq. \ref{hiperplane}.

\begin{eqnarray}
\vec{\omega}^{T} \textbf{R}(\vec{\mu}) = 0
\label{hiperplane}
\end{eqnarray} 

Euclidean distances measurements $\delta$ of each point $\in$ $\textbf{R}_{1}$ and $\textbf{R}_{2}$ to the separator hyperplane brings out the information of how much classifiable a shale ($\textbf{R}_{1}$) is from the others groups of rocks ($\textbf{R}_{2}$). The relation of all distance measurements  is decried in Eq. \ref{ISL}. The first parcel is the sum of all point measurements related to $\textbf{R}_{1}$ class. The second parcel is the sum of all point measurements related to $\textbf{R}_{2}$ class.

\begin{eqnarray}
\Upsilon =\left|\sum_{i=1}^{n}\frac{\delta_{i}}{|\delta_{i}|}\right| + \left|\sum_{j=1}^{m} \frac{\delta_{j}}{|\delta_{j}|}\right| 
\label{ISL}
\end{eqnarray}
\begin{eqnarray}
i=1,2,...,n \nonumber
\\
j=1,2,...,n \nonumber
\end{eqnarray}

Where $\Upsilon$ is the Linear Separable Index. 

Table 1 shows all percentage $\Upsilon$ values for each lithotype class that are present in well T$1$, Fig.\ref{t1}.

%\tabl{example}{This table shows the result of linear discrimination analysis for all the nine lithotype presents in the training well T1.}
{
	\begin{center}
		
		\begin{tabular}{|c|c|}
			\hline
			\multicolumn{2}{|c|}{Linear Discriminator Analysis ($\Upsilon$)} \\
			\hline
			Shale &  62.5 \%\\
			\hline
			Dolomite & 62.5\% \\
			\hline
			Diabase   & 75\%  \\
			\hline
			Conglomerate   & 62.5\%  \\
			\hline
			Conglomerate + Basement  1 & 50\%  \\
			\hline
			Conglomerate + Basement  2 & 50\%  \\
			\hline
			Conglomerate + Basement  3 & 50\%  \\
			\hline
			Conglomerate + Basement  4 & 50\%  \\
			\hline
			Basement   & 37.5\%  \\
			\hline
			
		\end{tabular}
	\end{center}
	\label{tab}
} 

\subsection{Perceptron - SLP}

The single-layer perceptron (SLP) is basic component in the context of multilayer feed-forward networks used to solve real-world problems. The behavior of a single neurone serves as the simplest prototype model for studying the characteristics of more general non-linear models such as multilayer perceptron \citep{Raudys1998}.

The mathematical model of a single Perceptron is divided in two parts. First is training where the inputs $\left\lbrace \xi_{i}  \middle| i = 1, 2, 3, 4, \dots, n \right\rbrace$ ($\textbf{R}_{1}$ class) are related to $\left\lbrace \omega_{i}  \middle| i = 1, 2, 3, 4, \dots, n \right\rbrace$ (weights) according to the \textit{net} function described in Eq. \ref{net}. 


\begin{center}
	\begin{tikzpicture}
	\node[functions] (center) {};
	\node[below of=center,font=\scriptsize,text width=4em] {Threshold};
	\draw[thick] (0.5em,0.5em) -- (0,0.5em) -- (0,-0.5em) -- (-0.5em,-0.5em);
	\draw (0em,0.75em) -- (0em,-0.75em);
	\draw (0.75em,0em) -- (-0.75em,0em);
	\node[right of=center] (right) {};
	\path[draw,->] (center) -- (right);9
	\node[functions,left=3em of center] (left) {$\sum$ net};
	\path[draw,->] (left) -- (center);
	\node[weights,left=3em of left] (2) {$\omega_{3}$} -- (2) node[input,left of=2] (l2) {$\xi_{3}$};
	\path[draw,->] (l2) -- (2);
	\path[draw,->] (2) -- (left);
	\node[below of=2] (dots) {$\vdots$} -- (dots) node[left of=dots] (ldots) {$\vdots$};
	\node[weights,below of=dots] (n) {$\omega_{n}$} -- (n) node[input,left of=n] (ln) {$\xi_{n}$};
	\path[draw,->] (ln) -- (n);
	\path[draw,->] (n) -- (left);
	\node[weights,above of=2] (1) {$\omega_{2}$} -- (1) node[input,left of=1] (l1) {$\xi_{2}$};
	\path[draw,->] (l1) -- (1);
	\path[draw,->] (1) -- (left);
	\node[weights,above of=1] (0) {$\omega_{1}$} -- (0) node[input,left of=0] (l0) {$\xi_{1}$};
	\path[draw,->] (l0) -- (0);
	\path[draw,->] (0) -- (left);
	\node[below of=ln,font=\scriptsize] {inputs};
	\node[below of=n,font=\scriptsize] {weights};
	\end{tikzpicture}
\end{center}

\begin{eqnarray}
\textit{net}=\sum_{i=1}^{n} \xi_{i} \cdot \omega_{i} + \beta_{i}
\label{net}
\end{eqnarray}

where $\beta_{i}$ is bias\footnote{Bias is a tendency in which is possible to move the separator hyperplane. This work considered $\beta_{i}=0$}. In this part the algorithm output are the news $\omega_{i}$ net.

The second part is the classification where the weights calculated in the previous phase are used togheter with a new set of inputs (classification well C$1$) and an activation function (sign function) Eq.\ref{sign} .

\begin{eqnarray}
\Theta(net) & = 
\begin{cases}
+1 ; net > 0 \\
-1 ; net \leq 0
\end{cases}
\label{sign}
\end{eqnarray}

Perceptron's classification output is described as Eq \ref{percep}.

\begin{equation}
\textit{S} :=  \Theta  (\sum_{i=1}^{n} \mu_{i} \cdot \omega_{i} + \beta_{i})
\label{percep}
\end{equation}

where $\mu_{i}$ are the classification set input (C$1$ well), and $\textbf{S}$ are the classified shale lithotype rock.

\subsection{Results}

This proposed methodology infers the localization of only one type of rock. Shale is the chosen target for his experiment. Distances between shale and hyperplane projection resulted of the LDA analysis is presented in Fig. \ref{LDA}. 

\begin{figure}[ht]
	\centering % para centralizarmos a figura
	\includegraphics[scale=0.2]{Images/LDA-folhelho.png} 
	\caption{Linear discrimination based on euclidean distances $\delta_{i}$. Green circles shows the shales distances relative to the projection of the hyperplane (black line). Red circles represents all 9 lithotype distance present in training well T$1$.}
	\label{LDA}
\end{figure}

Linear discriminant analysis shows that training data number $2$, $3$ and $5$ were not correctly separated by LDA analysis. That's indicates that $62.5\%$ of shale class can be identification by Perceptron model. 

Fig. \ref{classification} shows results of the perceptron analysis for shale search on C$1$ synthetic data. 


\begin{figure}[ht]
	\centering % para centralizarmos a figura
	\includegraphics[scale=0.5]{Images/shaleclass.png} 
	\caption{Classification for shale analysis. (a) original C$1$ well. (b) Perceptron classification for shale.}
	\label{classification}
\end{figure}

Perceptron shale's classification show that a linear approach don't recognize the bottom layer of shale. Dolomite-shale contact was not identifiable.

\subsection{Conclusion} 

This methodology intent to indicate which rock are eligible for a classification by a linear model such SLP. In a synthetic scenario $\Upsilon$ index indicates that shale and diabase are candidates for classification by a linear feed-forward model (Table 1).

LDA analysis indicates that shale will just recognizable in $62.5\%$ of the cases, based on training data (Fig. \ref{LDA}). This is explained by the physical properties that have been used for defining lithology in the model. Concerning resistivity shale have a broad scale measurements in which could affect directly the shale-dolomite contact.

Perceptron classification have direct connection with LDA analysis. Fig. \ref{classification} shows that perceptron (a linear separator model) could not separate the base contact between dolomite and shale. One possible reason is related to resistivity measurements. Electrical data changes a lot under the presence of water or oil content inside the rock pores. This wide range of values prevented perceptron of finding this rock base contact.


\section*{Acknowledgements}
Thanks to CNPq scientific Brazilian research foundation. 

\begin{figure}[H]
	%\onecolumn
	\centering % para centralizarmos a figura
	\includegraphics[scale=0.6]{Images/Pocot1.pdf}
	\caption{Synthetic Sedimentary well T1 used for training (based on \citet{Sal2008}) }
	\label{t1}
\end{figure}

  \bibliography{references.bib}
  
\end{document} 
